\documentclass[12pt,onecolumn]{article}

\begin{document}

\title{Automatic Navigation Mesh and Link Generation using Onion BSPs}
\author{Stuart Golodetz}
\date{}
\maketitle

\section*{Introduction}

The representation of the walkable area of a 3D environment in such a way as to facilitate successful navigation by intelligent agents is an important problem in fields such as computer games \cite{?}, architectural walkthrough \cite{?} and robot navigation \cite{?}, and it has been extensively studied. As surveyed by Tozour \cite{tozour04}, there are a variety of common ways to represent such an environment: TODO

Since their introduction by Greg Snook in 2000 \cite{snook00}, navigation meshes have proved to be a particularly successful representation due to TODO. As a result, they have seen heavy use in popular games such as TODO, and many games authors have contributed to their theoretical development \cite{?}. There has also been significant interest from academia, most notably TODO.

One facet of using navigation meshes is how to build them in the first place, and various methods have been described in the literature. TODO: Survey the various methods.

In this article, I describe my implementation of navigation mesh construction in my homemade \emph{hesperus} engine \cite{?}, based heavily on the approach of van Waveren in \cite{?}. The goal is to provide a helpful, implementation-focused introduction for those with no prior experience in the area. TODO: Give an overview of the method.

\section*{Onion BSPs}

TODO

\section*{Mesh Generation}

TODO

\section*{Walk and Step Link Generation}

TODO: Edge plane table construction, intervals, etc.

\section*{Localisation and Movement}

TODO

\section*{Potential Extensions}

\subsection*{Crouch Links}

TODO

\subsection*{Ladder Links}

TODO

\subsection*{Jump Links}

TODO

\section*{Evaluation}

TODO

\section*{Conclusions}

TODO

\section*{Acknowledgements}

TODO

\nocite{*}

\bibliographystyle{plain}
\bibliography{navmesh}

\end{document}